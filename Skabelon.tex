\documentclass[a4paper]{article}
\usepackage[utf8]{inputenc}
\usepackage[danish]{babel}

\usepackage{hyperref}
\usepackage{amsmath}
\usepackage{amsfonts}
\usepackage{amssymb}
\usepackage{graphicx}
\usepackage{fancyhdr}
\usepackage{moreverb}

% Ved at bruge kommandoen \newcommand kan man forkorte kommandoer eller ændre dem til noget mere passende.
\newcommand{\setR}{\mathbb{R}}
\newcommand{\setZ}{\mathbb{Z}}
\newcommand{\setN}{\mathbb{N}}
\newcommand{\setF}{\mathbb{F}}
\newcommand{\lra}{\leftrightarrow}
\newcommand{\Lra}{\Leftrightarrow}
\newcommand{\ra}{\rightarrow}
\newcommand{\Ra}{\Rightarrow}
\newcommand{\ac}{\textasciicircum}
\newcommand{\uuline}[1]{\underline{\underline{#1}}}
\newcommand{\bpm}{\begin{pmatrix}}
\newcommand{\epm}{\end{pmatrix}}

\renewcommand{\headrulewidth}{0pt}
\title{Delrapport 2}
\author{Anders Brandhof '190493' SGL135 \\ Andreas Jørgensen '240594' SRV415 \\ Casper Iversen '090691' JVP497 \\ Søren Jensen '270792' PWS412 \\
Instruktor: Markus Lund W\\
ProjDat2015}
\begin{document}

\maketitle

\pagebreak

\tableofcontents

\newpage

\section{Abstract}
Københavns Erhvervs Akademi has requested a better solution to loan computers and IT-equipment to their employees and students. Their current system is based on paper and is very time consuming. They are having trouble archiving them, and at the same time some of the papers have disappeared using this system. Our customer has therefore requested an IT-solution to store the information of the loaner computers. We are making a library lending system for their computers and IT-equipment, which will be used by their employees in their IT-department. We will create a database, which will store all the information. The information is: Their users, their computers and if a computer is available or not. Our backend will be made in Python. Our costumer has requested us to create a front-end to our liking. \\
\section{Formål og rammer}
Formålet ved projektet er, at vi skal lave et program eller website opsat udlåns-system. Dette system skal have til formål at holde opsyn med hvilke computere der er udlånt til hvilke personer og hvilke computere der er til rådighed til at blive udlånt til ansatte eller studerende i deres database.\\
Formålet ved projektet er, at vi skal lave et program eller website opsat udlåns-system. Dette system skal have til formål at holde opsyn med hvilke computere der er udlånt til hvilke personer og hvilke computere der er til rådighed til at blive udlånt til ansatte eller studerende i deres database.\\
Måden vi vil gøre dette på er at lave en database der skal stå for opmagasineringen af informationerne om både computerne men også hvem der har dem eller om de er klar til at blive udlånt.\\
Udover dette skal lave vi et program til at administrere data’en fra databasen og formidle den på en fornuftig måde overfor brugeren.\\
Vi skal til slut lave et program eller website der så fremviser dette på en nem forståelig måde til brugeren så de kan indtaste hvilke computere der er udlånt til hvem og hvilke computere der er klar til udlåning eller hvilke computere der bliver afleveret.\\ \\
\subsection{FACTOR Analyse}
\textbf{Functionality:} Programmet administrere udlån af computere i form af hvem har en computer, er en computer udlånt, og om en computer er klar til udlån samt om hvor længe en person har haft en given computer.\\
\textbf{Application domain:} For at løse problem domain som er at få et bedre system end at skulle skrive manuelt ned på papir hver gang en computer skal udlånes, så skal programmet håndtere en bruger der skal låne computere ud til lånere. Dette vil programmet gøre ved hjælp af en database som har information om låneren og hvilke computere der er klar til at blive udlånt.\\
Systemet skal udover dette også håndtere at sende en rykker ud til alle de personer som der har haft computerne i for lang tid, derfor har databasen også et timestamp der viser hvornår computeren blev udlånt og hvornår den bliver afleveret.\\
\textbf{Conditions:} Systemet vil blive kodet i Python og vil være begrænset således at det kun er udlåneren som har adgang til systemet, men en låner er nød til at kontakte udlåneren for at låne en computer.\\
\textbf{Technology:} Systemet vil blive kodet i Python og vil køre over python driveren ligeledes.\\
Systemet skal være tilgængeligt til at køre på både pc og mac, og vil komme til at have en database som vil ligge på kundens drev som bliver delt ud til kundens ansatte således at alle vil kunne bruge det derfra.\\
\\
\textbf{Objects:} Udlåner og låner. computere og måske andre elektroniske devices.\\
\textbf{Responsibility:} Systemets ansvar ligger i, at den skal udskifte skriftlige udlånssedler, som fungerer som værdipapirer hos kunden. Dette system skal være et mere sikkert system, hvor data/sedler ikke vil kunne forsvinde, som det gør for dem nu.
\section{Kravspecifikationer}
\subsection{Funktionelle krav}
Funktionelle krav beskriver samspillet mellem systemet og dets omgivelser uafhængigt af dens implementation. Miljøet omfatter brugeren og ethvert andet eksternt system, som systemet interagerer med. \cite{OOSE}. De funktionelle krav beskriver altså, hvordan system skal køre. Vi har altså en liste af de ting, som er forventet systemet skal kunne.\\
$\bullet$ Systemet skal kunne skanne en stregkode og modtage et brugerID som man indtaster manuelt. \\
$\bullet$ Systemet skal kunne modtage og opbevare en underskrift. \\
$\bullet$ Systemet skal kunne udskrive/sende kvittering til låneren \\
$\bullet$ Systemet skal kunne udlåne og aflevere datamater. \\
$\bullet$ Systemet skal kunne vise data for hhv. lånerne og datamater. \\
$\bullet$ Systemet skal kunne have mulighed for at redigere lånerstatus og computerstatus i databasen af en bruger af systemet. \\
$\bullet$ Systemet skal have en hovedeside, hvor brugeren kan tilgå de forskellige funktioner.\\
\subsection{Ikke funktionelle krav}
Ikke-funktionelle krav beskriver aspekter af systemet, som ikke er direkte relateret til den funktionelle opførsel af systemet. Ikke-funktionelle krav omfatter en bred vifte af krav, der gælder for mange forskellige aspekter af systemet, fra usability til performance. \cite{OOSE}{119} \\ De ikke funktionelle krav, er lidt omvendt i forhold til de hvad funktionelle krav beskriver. De ikke funktionelle krav beskriver altså ikke hvad systemet skal kunne, men hvordan det skal kunne køre. Det er altså kravne til det miljø, som systemet skal køre i. \\
$\bullet$ Systemet skal kunne tilgås fra Windows 7 og 8. \\
$\bullet$ Tilgængeligheden på programmet skal være enten via en hjemmeside eller via et fælles drev, dette vil blive uddybet efter vores møde. \\
$\bullet$ Programmet skal være sikkert, altså skal det ikke kunne hackes eller kunne tilgås udefra, altså skal man være forbundet til KEA's netværk. Det skal også være sikkert ved, at der ikke mistes data i systemet. \\
$\bullet$ Programmet må ikke køre langsomt, da. der i forvejen er for mange it-systemer kunden bruger der kører for langsomt. \\
$\bullet$ Systemet må ikke være case-sensitive. Systemet må altså ikke skelne mellem store og små bogstaver ved inputs. Dette er for at undgå brugerfejl. 
\newpage
\subsection{Use-Case-model}
Use cases beskriver opførelsen af systemet, som set fra en Actor’s synspunkt. En Actor er en extern entity som interagere med systemet. En use case beskriver en funktion i systemet som en del af flere events, som giver Actor et visuelt resultat. En Actor starter en use case når de tilgår systemets funktionalitet. En use case kan starte en ny use case, og derfra hente mere information fra Actor.\cite{OOSE}{42} \\ \\
Vi har lavet følgende udkast, som er et udtryk for, hvordan vi forventer vores system vil køre:\\ \\
\begin{figure}[h!]
\includegraphics[width=0.8\textwidth]{UseCaseModel}
  \caption{Use-Case-Model}
  \centering
\end{figure}  \\ \\
Figur 1 illusterer brugerens råderum. På figuren ses det, at brugeren skal have mulighed for at udlåne, slette, aflevere, redigere og tilføje IT-udstyr, her primært computere. Låneren har mulighed for at igang sætte et lån, eller aflevere en computer. Dette kan de gøre ved at henvende sig til en udlåner. Brugerne af vores system, skal have adgang til alle systemets funktioner. Der bliver derfor ikke skelnet i brugerrettigheder. \\
\textbf{Use-case}\\
Følgende use-cases er udarbejdet for at beskrive det nye, og det gamle system. Deres mening er at kaste lys over fordelene ved det nye system, såvel som bagdelene ved det gamle system. \\
\begin{table}[h]
\caption{Use-case 1}
\begin{tabular}{ll}
Use case name               & \underline{GammeltUdlånssystem} \\ \hline
Participating actor           & \underline{Låner} \\
instances                     & \underline{Udlåner} \\ \hline
Flow of evnets                & 1. Låneren mangler en computer.\\& Låner henvender sig derfor til udlåneren for at låne en.
\\& 2. Udlåneren tjekker i arkivet, om der er en ledig \\& computer. Udlåner finder en.
\\& 3. Udlåneren giver låneren en seddel, som låneren skal udfylde: \\& Computer Mærke og model, S/N Tyveri-ID, \\& Låner, Navn, KEA login, Afdeling, Telefon/mobil, \\& Bemærkninger Underskrift, Låners underskrift \\& KEA Servicedesk Sagsnr, Udlånt af, Dato.
\\& 4. Udlåneren henter computeren til låneren og arkiverer
\\& seddelen.
\end{tabular}
\end{table} \\
Dette gamle udlånssytem kræver ekstra unædvendig tid fra både låner og udlåner. Udlåner riskere også, at der går rod i papirarbejdet og at tingene forsvinder. Systemet er altså forældet, og derfor ønsker kunden en løsning istedet. \\
\begin{table}[h]
\caption{Use-case 2}
\begin{tabular}{ll}
Use case name             & \underline{NytUdlånssystem-Udlån} \\ \hline
Participating actor           & \underline{Låner} \\
instances                     & \underline{Udlåner}\\ \hline
Flow of evnets                & 1. Låner mangler en computer.	\\& Låner henvender sig derfor til udlåneren for at låne en.
\\& 2. Udlåneren tjekker i arkivet, om der er en ledig \\& computer. Låner finder en.
\\& 3. Udlåneren indtaster lånerens ID og skanner computeren.
\\& 4. Udlåneren henter computeren til låneren.
\end{tabular}
\end{table}\\
Her er samme process beskrevet, blot med brug af det nye udlånssystem. Her er det tydeligt at processen er meget hurtigere og nemmere, da punkter som Mærke/model, S/N Tyveri-ID, og afdeling, etc. udfyldes automatisk. Derudover er det ikke nødvendigt at arkivere fysiske kvitteringer, der har risiko for at gå tabt.
\newpage
\begin{table}[h]
\caption{Use-case 3}
\begin{tabular}{ll}
Use case name               & \underline{NytUdlånssystem-Aflevering} \\ \hline
Participating actor           & \underline{Låner} \\
instances                     & \underline{Udlåner}\\ \hline
Flow of evnets                & 1. Låner skal aflevere en computer, som har været udlånt.
\\& Låner henvender sig derfor til udlåneren for at aflevere den.
\\& 2. Udlåneren indtaster lånerens brugerID og skanner computeren.
\\& Udlåneren arkiverer computeren.
\end{tabular}
\end{table}
Til sidst er afleveringen af lånerens computer beskrevet. Som det ses er dette også hurtigt og let, da den nye afleveringsmetode, ligesom udlåningsmetoden, ikke producerer fysiske værdipapirer, der kan gå tabt. Derudover udfyldes felter som Mærke/model, S/N Tyveri-ID, og afdeling, etc. igen automatisk. \\
\subsection{Klassediagram}
Klassediagrammer bruges til at beskrive strukturen af systemet. Klasser er abstraktioner, der angiver struktur og adfærd af et sæt objekter. Objekter er instanser af klasser, der oprettes, ændres og ødelægges under udførelsen af systemet. Klassediagrammer beskriver systemet i form af objekter, klasser, attributter, operationer, og deres sammenslutninger. \cite{OOSE}{30}
\subsection{BCE-Model}
Entity objekter repræsenterer vedvarende oplysninger der registreres af systemet. Boundary objekter repræsenterer samspillet mellem aktører og systemet. Control objekter er ansvarlig for at realisere use cases. Modellering af systemet med Entity, Boundary og Control objekter giver udviklere enkel heuristik til at skelne mellem forskellige, men beslægtede begreber.\cite{OOSE}{171}
\begin{figure}[h!]
\includegraphics[width=0.9\textwidth]{BCE.png}
  \caption{BCE-Model}
  \centering
\end{figure} \\ \\
I BCE-modellen ses det hvordan udlåneren kan interagere med interfacet. Gennem dette får han/hun dagang til funktioner, såsom udlånComputer, afleverComputer, historik, og tilføj computer. Alle disse funktioner har tildelt en metode, der udfører udlånerens hensigter.\\
\subsection{Sekvensdiagram}
Et sekvens diagram forbinder use cases med objekter. Det viser, hvordan en use case (eller scenarier) opfører sig og hvordan det fordeles blandt de deltagende objekter. Sekvens diagrammer er normalt ikke så godt et medium for kommunikation med brugeren som use cases er, da sekvens diagrammer kræver mere baggrund om notation. \cite{OOSE}{179} \\
\begin{figure}[h!]
\includegraphics[width=0.9\textwidth]{Sekvensdiagram2.png}
  \caption{Sekvensdiagram-1}
  \centering
\end{figure}
På ovenstående sekvensdiagram vises hvordan det nye udlånssystem kommer til at fungere ved udlån af computer. \\
\newpage
\begin{figure}[h!]
\includegraphics[width=0.9\textwidth]{Sekvensdiagram3.png}
  \caption{Sekvensdiagram-2}
  \centering
\end{figure}
På ovenstående sekvensdiagram vises hvordan det nye udlånssystem kommer til at fungere ved aflevering af computer. Som det ses er processerne for denne og udlån af computer meget ens.
\newpage
\section{Systemdesign}
Vi har lavet et deployment diagram, som viser systemets opbygning. Deployment diagrammet bliver brugt til at vise rune-time komponenter og nodes. \cite{OOSE}{256}
\begin{figure}[h!]
\includegraphics[width=0.9\textwidth]{deploymentdiagram.png}
  \caption{Deployment Diagram}
  \centering
\end{figure} \\ \\
Systemet er bygget op om en database. Databasens struktur er blevet udviklet, så det er nemt for os at udvinde den information vi eller kunden ønsker. Kravene til databasen er, at den skal opbevare data for brugerne, computerne, udlån og historikken for de udlånte computere. Vi lavede en tabel for hver af disse. Vi kom frem til den konklusion, at det ikke var nemt at udvinde information om hvilke computere var ledige, uden at man skulle slette noget af sin data i en tabellen. Dette var fordi, at når en Computer var blevet afleveret, da vil dens data stadig være i en tabel for de udlånte computere, alle afleverede computere ville altså stå som værende udlånt. Strukturen er blevet ændret til kun at indholde brugere, computere og lån, hvor historikken findes i lån. \\ \\
Vi har i gruppen diskuteret vores forskellige muligheder for backend:
Vi har indtryk af, at Python er en af de mest brugte sprog til databaser. Vi alle i gruppen tidligere arbejde i Python i blok 3. Dette var i forbindelse med eksamen i Datamining. Alle i gruppen har derfor kendskab til Python.\\ \\
Vi har i gruppen overvejet C++. C++ opfører sig på nogle punkter ligesom java, da det er objekt-orienteret, hvilket vil sige, at det vil gøre lærings processen nemmere for os. Ud fra det overblik vi har dannet, da er det meget nemt at finde information om. Her er det primært information implimentation af databaser, men også stortset alt hvad vi kan have brug for. Dog er ulempen, at ingen kender dette sprog eller har erfaringer med det.\\ \\ \\
Vi er kommet frem til, at programmet vil blive skrevet i python, vi besluttede dette da vi følte vi havde mest kendskab til python og vil kunne bruge det på bedre vis end andre sprog. \\
Vi har derfor lavet vores backend i Python, hvor vi har lavet et program, som kan hente det data vi har brug for fra databasen. Vores Python program anvender sqlite3 biblioteket. \\
Vi har ikke lavet brugerfladen endnu. Dette vil blive gjort i enten python som et program eller blive en hjemmeside lavet i Django.\\ Grunden til hjemmesiden vil være i Django, er fordi det er et python framework. Vi har været i kontakt med kunden omkring det skulle være et program eller en hjemmeside. Til dette havde kunden ingen krav, og lader det være op til os. Vi er stadig i processen, hvor vi diskuterer de forskellige fordele og ulemper. En af fordelene ved at lave et python program er, at vi ikke har behov for at lære et nyt sprog. En fordel ved at lave en hjemmeside er, at der kan laves en pænere brugerflade, men det er et helt nyt sprog som vi skal lære. \\
Vores kunde har derudover også stillet os det krav, at vi skal sikre os at der kan indsættes en underskrift i systemet. Vi har endnu ikke bestemt os for hvordan dette skal gøres. Vi skal enten få underskriften fra en touchpad/mus på en computer, eller koble en tablet til computeren og få brugeren til at skrive sin underskrift på denne. Vi har da tænkt os at opbevare billedet med underskriften i en mappe på et fællesdrev, hvor vi tilknytte de enkelte underskrifter til de der tilhørende udlån. \\ \\
Udestående opgaver, som mangler at bliver lavet: \\
$\bullet$ Front-end. \\
$\bullet$ Implementation af underskrift. \\
$\bullet$ Program som udskriver kvittering, når en computer afleveres.
\section{Program- og systemtest}
Vi har tænkt os at dele program- og systemtest op i to. Vi vil her teste brugervenligheden og om systemet virker som det skal. Vores plan for at teste brugervenligheden er, at vi vil lave Think-Aloud test med vores kunde. Vi vil her observere, hvordan vores kunde færdes i vores system og give vores kunde flere forskellige opgaver som skal løses ved brug af vores system. Vi vil her undervejs notere de gode og dårlige ting ved systemet. Når testen med kunden er færdig, vil vi evaluere testen med kunden, og høre hvad kunden synes om systemet, eller om noget er uhensigtsmæssigt. Think Aloud testen vil blive lavet i forhold til artiklen "Thinking Aloud - User Testing by Molich". \\ \\
Vi vil teste vores system, for at se om det virker som det er tiltænkt. Vi vil altså lave test cases for vores funktioner i python. Vi vil her f.eks. teste, om det er muligt at låne den samme computer ud til to personer på samme tid, om det er muligt at udlåne en fri computer som har været udlånt, men er blevet afleveret. Vi vil altså teste alle de essentielle funktioner, som får systemet til at fungere. Vi vil her teste systemet i forhold til de metoder, som er nævnt i artiklen "Sestoft "Systematic Software Testing" 2008".
\section{Brugergrænseflade og interaktionsdesign}
Ikke implementeret endnu.\\
\includegraphics[scale=0.07]{1.jpg}\\
GUI 1\\
\includegraphics[scale=0.07]{4.jpg}\\
GUI 2\\
\includegraphics[scale=0.07]{5.jpg}\\
GUI 3\\
\includegraphics[scale=0.07]{6.jpg}\\
GUI 4\\
\includegraphics[scale=0.07]{7.jpg}\\
GUI 5\\
\includegraphics[scale=0.07]{8.jpg}\\
GUI 6\\\\
På de overnstående billeder GUI 1-6 ses de ideer vi har til hvordan det kunne komme til at se ud.\\
Dette er dog stadig kun ide face af hvordan det muligtvis kunne komme til at se ud, derfor har vi en masse forskellige GUI ideer som vi kan se forskellige gode aspekter ved.\\
Disse billeder er en sammensætning for at se et samlet GUI således at GUI 1 og 5 udgøre vores første GUI med en drop down menu som skal give et bedre overblik over de forskellige handlinger.\\
GUI 2,3,4,6 udgøre vores andet GUI design som fremstår på en mulig hjemmeside version hvor vi har menuerne i siden og kan komme ind på de forskellige sektioner som vist på billederne.
\pagebreak
\section{Projektsamarbejdet}
Når vi ser på samarbejdet mellem gruppen og kunden har vi haft en del problemer i forstanden at vi ikke har haft mulighed for at få et møde i før for få dage siden. Dette har vi da fået løst da vi formåede at presse på og få et møde og få langt støre klarhed omkring projektet. Før dette havde vi blot en email at arbejde ud fra dog gav den også godt udgangspunkt for hvad vi kunne starte med at arbejde med.\\
Vi har efter dette haft bedre håndtering af arbejdet da vi har fået langt større klarhed omkring projektet. Dog har kunden til dette møde kommet med et meget pludseligt krav til projektet som vil tilføre en større mængde arbejde til projektet end forventet. Kravet var at vi skulle tilføje et underskrift system af en art til hvad vi allerede havde fået information om at lave. Dette har vi haft forskellige forslag til hvordan vi skulle løse som f.eks. Have en tablet hvor personen skriver underskrift på og gemme billedet på samme drev og med navngivning med et sagsnummer så det kan findes via databasen.\\
Anden forslag ville kunne være at lave det hele over en hjemmeside og kode noget javascribt ind og sætte et viatuelt underskrift ind under også gemme den som et billede og arkivere på samme måde som nævnt i forslaget ovenover.\\
For at effektivisere vores arbejde med projektet, har vi indført to faste dage hver uge, hvor vi vil sidde og udvikle IT-systemet. Dette er for at sikre, at vi kan leve op til vores egne deadlines som vi (dog kun har sat som et personligt mål da kunden ikke har en fast deadline), og nå at få den krævende viden for at få et virkende program.\\
Vi er begyndt her at lave et startmøde i gruppen, når vi mødes, hvor vi uddelegerer arbejde og sikrer os, at vi alle arbejder på relevante felter.\\
Vi vil efter gruppearbejdet opdatere hinanden på hvad og hvor meget der er blevet lavet. Vi vil gøre det i form af forklaring af hver persons individuelle arbejde både i form af kode eller i form af hvilken strukturering af arbejdet vi hver har lavet.\\
Vi mener indtil videre ikke, at vores arbejdsindsats har været god nok, da vi har haft en del eksamener og reeksamener og derfor været nød til at udsætte arbejdet på projektet. det er derfor at vi har valgt, at ændre i vores egen møde struktur og indføre to faste dage om ugen.
\newpage
\section{Bilag 3 - Timeline}
På nuværende tidspunkt har vores arbejde ikke været på selve programmet endnu men snarere på ideen bag den, så vi vil have en bedre forståelse af hvilket produkt vi skal ende med. Dog er vores egne deadlines sat således:\\
Prototype klar den 03/05/2015 \\
Test af prototype med kunden den 04/05/2015\\
Vores timeline således ud:\\
$\bullet$ Mail modtaget fra kunde 25-03-2015:\\
Den 25. marts modtog vi mail fra kunden om hvad de ønskede sig af programmet. Kunden ønsker at få mere styr på deres proces og få elimineret deres nuværende papirseddel løsning. De ønsker, at systemet automatisk sender rykkere, men samtidig, at de også manuelt kan dette. Kunden vil gerne have, at systemet kan betjenes ved brug af en håndskanner.\\
$\bullet$ Afholdt gruppemøde 21-04-2015:\\
Den 21. april har vi aftalt, at vi ønsker at have den første prototype klar inden den 3. maj. Vi har endvidere aftalt at have to faste dage om ugen, hvor vi vil arbejde på projektet. Dette valg er truffet for at sikre os, at vores egen deadline vil kunne blive overholdt.\\
\pagebreak
\includegraphics[scale=1]{Referat.png} 
\section{Reviews}
\subsection{A Rational Design Process: How and Why to Fake It}
\textbf{Hovedpunkter}\\
Den rationelle tankegang efterstræbes i artiklen, men ikke den traditionelle form af slagsen som titlen giver udtryk for.
Der henvises til en løbende udviklingsform som der ses indenfor matematikken.
F. eks. bliver et matematikbevis ikke uddybbet på den måde i starten. Det starter med en ide som der derefter arbejdes videre på løbende. Indtil der i sidste ende står et "forhåbentligvis" meget skarpere og kortere bevis end den oprindelige ide var.
Denne form for tankegang kan der draves direkte paralleller til i forhold til programmering.\\
Den rationelle metode vil aldrig kunne blive hundrede procent realiseret indenfor programmeringens verden da usikkerheder og uforudsete problemer ikke kendes på forhånd. Dertil hører mange argumenter som omhandler menneskelige fejl, mangel på specifikke/ændrende krav fra kunde, dele af ældre projekter ville måske kunne kobles på det nye osv.\\
Længere fremme i artiklen kommer det store spørgsmål på, hvorfor denne dokumentation er nødvendig før, under og efter projektet. God dokumentation for det igangværende projekt kan være altafgørende for programmøre der kommer til undervejs i et projekt og fungere som dokumentation af processen så kunde/observatører kan se fremgangen.\\\\
\textbf{Relationer og Tanker}\\
Måden som artiklen beskriver et udviklingsforløb af et projekt ligner meget Scrum metoden.~\cite{Scrum} Især i forhold til den løbende udvikling af produktet. De runs de foregår hvorved løbende prototyper/opgaver skal nåes giver god mening i forhold til denne refleksive/løbende proces af projektet.\\
Kontra bogen~\cite{OOSE} bliver der i denne artikel reflekteret meget over hvad der bliver sagt, vi vil gå så langt som at sige, at teksten er direkte kritisk overfor sig selv. Forstået i den forstand at to sider af en sag bliver taget op. Det ser man for eksempel i artiklen da den fortæller om hvorfor vi skal bruge denne rationelle desgin process, men så alligevel ikke. Bogen er meget "one-sided" i forhold til mange ting, blandt andet denne design process hvor man ikke ser denne reflektion over processen men i stedet fokuserer på at forklare de forskellige metoder.\\
\pagebreak
\subsection{Designing for Usability: Key Principles and What Designers Think}
\textbf{Hovedpunkter}\\
Artiklen sætter fokus på selve bruger-testingen. Der indikeres at for lidt energi bliver lagt i flade udtryk såsom "brugervenlighed", "nemt interface" osv. Mens det der i virkeligheden er brug for er specifik testing af de personer som skal bruge programmet, f. eks. klienten. Artiklen omtaler dette som en meget overset aspekt af udviklingen af et projekt. Der menes at man ville kunne slippe for mange "dumme fejl". Fejl som brugere vil kunne komme ud for som måske ikke så intuitive for en programmør eller designer. Man kan sige, at selve brugervenligheden kan være udsat i større projekter fordi de anses som en udgift i stedet for en nødvendighed.\\
Her starter der en sektion om hvordan man eventuelt ville kunne teste før en eneste linje kode til selve programmet overhovedet er startet.\\\\
\textbf{Relationer og Tanker}\\
Som i den anden artikel~\cite{UseDesign} er der mange antagelser og forudanselser som grundlag for mange af argumenterne. I denne kommer det for eksempel gennem generaliseringen af af hvordan diverse projekter håndteres og hvorledes deres fokuspunkter bliver prioriteret.\\
Artiklen lægger fokus på iterativt design, brugervenlighed og funktonalitet som en vigtig del af et projekts start. Dette spænder godt overens med Scrum metoden~\cite{Scrum} fordi metoden er meget iterativ igennem sine runs hvori man udbygger sit program bid for bid. Undervejs i sådan et forløb får man prototyper/versioner af sit program man kan bruge som:\\
$\bullet$Præsentation til kunder af hvor langt i projektet man er nået og hvilken funkionalitet der er nået/mangler.\\
$\bullet$Løbende versioner som kan testes på brugervenlighed og funktionalitet.\\
$\bullet$Sidst men ikke mindst sikrer man sig at have tidligere versioner af programmet man kan gå tilbage til hvis noget skulle gå galt fra version til version.\\
\pagebreak
\bibliography{bibte}{}
\bibliographystyle{plain}

\end{document}
